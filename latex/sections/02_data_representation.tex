% PROJECT STATEMENT — relevant excerpts (see full text in `sections/project_statement_full.tex`)
% "adequate data representations should be pursued for projects with non-tabular data structures and unstructured variables using domain-guided transformations or representation learning (embeddings);"
% 
% "the underlying complexity of the datasets can be assessed via summarization capacity using linear methods, such as PCA, and non-linear methods, such as autoencoders;"
% 
% Use: justify chosen representation, show experiments (PCA variance explained, AE reconstructions), and record preprocessing steps applied to reach the representation.
\section{DATA REPRESENTATION}

\paragraph{Dataset (A) - AI News Analysis}
\subsection{Embedding Methodology}
Textual data was processed for embedding using the \texttt{Qwen2.5-8B-Instruct} model, generating high-dimensional vectors ($D = 4096$) that capture both structural and semantic nuances. 

Vector generation was orchestrated via the \texttt{vLLM} inference engine, optimized for Blackwell architecture with FP16 batching.
The resulting embeddings are persisted in a \texttt{PostgreSQL 16.x} instance equipped with the \texttt{pgvector} extension ($v0.7.0$).

\subsection{Dimensionality Reduction}
To mitigate the curse of dimensionality inherent to 4096-dimensional vectors and improve clustering density estimation, we employ \textbf{UMAP} (Uniform Manifold Approximation and Projection)~\cite{mcinnes2018umap}. The reduction is twofold:

\begin{enumerate}
    \item \textbf{Topological Projection ($D=5$)}: Vectors are reduced to 5 components to serve as the input space for density-based clustering (HDBSCAN). This dimension aligns with the intrinsic dimensionality of the dataset, estimated at $d \approx 4.11$ using the TwoNN algorithm~\cite{facco2017estimating}. This step counteract sparsity in high-dimensional space which degrades density estimation~\cite{mcinnes2018umap}, preserving local manifold structure while discarding noise.
    \item \textbf{Visual Projection ($D=2$)}: A further reduction to 2 components is generated strictly for visualization purposes, creating a "semantic topography" of the news landscape (Fig.~\ref{fig:umap_topography}).
\end{enumerate}

\begin{center}
    \includegraphics[width=\columnwidth]{images/images-clustering/theseus-topography-umap-base.png}
    \captionof{figure}{Semantic topography of the dataset(A) using UMAP (D=2).}
    \label{fig:umap_topography}
\end{center}

The UMAP manifold is constructed using the \textit{Cosine} metric with $n\_neighbors=15$ and $min\_dist=0.1$, ensuring that the angular relationships captured by the Qwen2.5 model are preserved in the lower-dimensional space.

\subsection{Data Transformations: LogProb Centrality}
To enrich the representation beyond raw embeddings, we apply a domain-guided transformation to assess the thematic centrality of each news item.

The centrality scoring employs a zero-shot generative prompt-weighting technique using the \texttt{Qwen2.5-7B-Instruct} model. Each news item is evaluated against the target theme to produce a continuous metric $S \in [0, 1]$ leveraging the \textbf{logprobs} extraction methodology as described in \textit{``The Mean-Difference''}~\cite{wallace2024meandifference}.

\begin{center}
    \includegraphics[width=\columnwidth]{images/images-clustering/theseus-topography-semantic-centrality.png}
    \captionof{figure}{Semantic topography overlaid with LogProb Centrality scores.}
    \label{fig:logprob_centrality}
\end{center}

\noindent As illustrated in Fig.~\ref{fig:logprob_centrality}, the dataset exhibits a strong semantic centrality towards the theme "Artificial Intelligence". The majority of the news items cluster in the high-confidence region ($S > 0.7$), validating the effectiveness of the initial keyword-based filtering while providing a nuanced gradient of relevance for items on the conceptual periphery.

This methodology was extended, but with \texttt{mesolitica/Qwen2.5-72B-Instruct-FP8} (because the scaling problem is more soft and needs a model with more semantic capacity), to map seven additional semantic dimensions of the news identity: \textit{Opportunity vs. Risk}, \textit{Regulatory Pressure}, \textit{Economic Momentum}, \textit{Ethics vs. Utility}, \textit{Technical Depth}, \textit{Geopolitical Scope}, and \textit{Urgency}. The integrated summary and complete distribution of these dimensions are available in Appendix~\ref{app:dna_summary} and \ref{app:dna_breakdown}.

\paragraph{Dataset (B) - Customer Personality Analysis}
\par \dotfillplaceholder
